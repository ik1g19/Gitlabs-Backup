%% %%%%%%%%%%%%%%%%%%%%%%%%%%%%%%%%%%%%%%%%%%%%%%%%
%% Problem Set/Assignment Template to be used by the
%% Food and Resource Economics Department - IFAS
%% University of Florida's graduates.
%% %%%%%%%%%%%%%%%%%%%%%%%%%%%%%%%%%%%%%%%%%%%%%%%%
%% Version 1.0 - November 2019
%% %%%%%%%%%%%%%%%%%%%%%%%%%%%%%%%%%%%%%%%%%%%%%%%%
%% Ariel Soto-Caro
%%  - asotocaro@ufl.edu
%%  - arielsotocaro@gmail.com
%% %%%%%%%%%%%%%%%%%%%%%%%%%%%%%%%%%%%%%%%%%%%%%%%%


\documentclass[12pt]{article}
\usepackage{design_ASC}
\usepackage{tikz}

\usepackage[utf8]{inputenc} % Required for inputting international characters
\usepackage[T1]{fontenc} % Output font encoding for international characters
\usepackage{stix} % Use the STIX fonts

\setlength\parindent{0pt} %% Do not touch this
%\setlength\parskip{1em}


\begin{document}


%%%%%%%%%%%%%%%%%%%%%%%%%%%%%%%%%%%%%%%%%
% Vertical Line Title Page
% LaTeX Template
% Version 2.0 (22/7/17)
%
% This template was downloaded from:
% http://www.LaTeXTemplates.com
%
% Original author:
% Peter Wilson (herries.press@earthlink.net) with modifications by:
% Vel (vel@latextemplates.com)
%%%%%%%%%%%%%%%%%%%%%%%%%%%%%%%%%%%%%%%%%

\begin{titlepage} % Suppresses displaying the page number on the title page and the subsequent page counts as page 1
	
	\raggedleft % Right align the title page
	
	\rule{1pt}{\textheight} % Vertical line
	\hspace{0.05\textwidth} % Whitespace between the vertical line and title page text
	\parbox[b]{0.75\textwidth}{ % Paragraph box for holding the title page text, adjust the width to move the title page left or right on the page
		
		{\Huge\bfseries Database Modelling}\\[2\baselineskip] % Title
		{\large\textit{COMP1204 - Coursework 2}}\\[4\baselineskip] % Subtitle or further description
		{\Large\textsc{Isaac Klugman}} % Author name, lower case for consistent small caps
		
		\vspace{0.5\textheight} % Whitespace between the title block and the publisher
		
		{\noindent Student ID: 30979536}\\[\baselineskip] % Publisher and logo
		{\noindent Username: ik1g19}\\[\baselineskip] % Publisher and logo
	}

\end{titlepage}


\pagebreak

\section*{The Relational Model}

\subsection*{EX1}

The relations and their relevant data types:

-$dateRep$ - The day, month and year in date format of when the record was collected. TEXT Data type.

-$day$ - The number of the day of the month of which the record was collected. INTEGER Data type.

-$month$ - The number of the month of which the record was collected. INTEGER Data type.

-$year$ - The year of which the record was collected. INTEGER Data type.

-$cases$ - The number of recorded cases on that day. INTEGER Data type.

-$deaths$ - The number of deaths on that day. INTEGER Data type.

-$countriesAndTerritories$ - The name of the country or territory. TEXT Data type.

-$geoId$ - The geographical identifier for that country or territory. TEXT Data type.

-$countryTerritoryCode$ - The code for that country or territory. TEXT Data type.

-$popData2018$ - The population of the country or territory as of 2018. INTEGER Data type.

-$continentExp$ - The name of the country or territory's continent. TEXT Data type.

\subsection*{EX2}

Functional dependencies in the Data Set:

-$dateRep \rightarrow day, month, year$

-$countriesAndTerritories \rightarrow geoId, countryterritoryCode, continentExp, popData2018$

-$geoId \rightarrow countriesAndTerritories, countryterritoryCode, continentExp, popData2018$

\vspace{1em}

Individually, $day, month, year, cases, deaths, countryTerritoryCode$ and $continentExp$ do not functionally determine anything.

In this given data set, $popData2018$ functionally determines $countriesAndTerritories, geoId, continentExp$ and $countryTerritoryCode$. However it is possible that two countries could of had the same population in 2018, if a country was added to this data set with an identical population in 2018 then $popData2018$ would no longer functionally determine any other attributes.

$countryTerritoryCode$ would ordinarily determine $countriesAndTerritories, geoId, continentExp$ and $popData2018$. However in the given data set, there are some instances where $countryTerritoryCode$ is blank, meaning that the code can no longer uniquely determine any other attributes.

\subsection*{EX3}

Potential Candidate keys:

-$dateRep, geoId$

-$dateRep, countriesAndTerritories$

-$dateRep, countryterritoryCode$

\subsection*{EX4}

$dateRep$ and $geoId$ make a suitable composite primary key.

$dateRep$ can uniquely identify a record for each country and $geoId$ is a short string. I am not using $countriesAndTerritories$ as the strings are longer and $countryterritoryCode$ is not present for all countries.

\section*{Normalisation}

\subsection*{EX5}

Existing partial key dependencies:

-$day, month, year \rightarrow dateRep$

-$countriesAndTerritories, countryterritoryCode, continentExp \rightarrow geoId$

\vspace{1em}

To solve this we can divide the original relation into three relations:

-$dateRep, day, month, year$

-$geoId, countriesAndTerritories, countryterritoryCode, continentExp, popData2018$

-$dateRep, geoId, cases, deaths$

\subsection*{EX6}

The three new relations:

\begin{center}
\begin{tabular}{ |c|c|c|c| } 
 \hline
 \multicolumn{4}{|c|}{Date} \\
 \hline
 dateRep & day & month & year \\
 \hline
 DATE & INTEGER & INTEGER & INTEGER \\
 \hline
\end{tabular}
\end{center}

\begin{center}
\begin{tabular}{ |c|c|c|c|c| } 
 \hline
 \multicolumn{5}{|c|}{Country} \\
 \hline
 geoId & countriesAndTerritories & countryterritoryCode & continentExp & popData2018 \\
 \hline
 TEXT & TEXT & TEXT & TEXT & INTEGER \\
 \hline
\end{tabular}
\end{center}

\begin{center}
\begin{tabular}{ |c|c|c|c| } 
 \hline
 \multicolumn{4}{|c|}{Virus Info} \\
 \hline
 dateRep & geoId & cases & deaths \\
 \hline
 TEXT & TEXT & INTEGER & INTEGER \\
 \hline
\end{tabular}
\end{center}

The primary keys for their respective tables are:

-Date - $dateRep$

-Country - $geoId$

-Virus - $dateRep, geoId$

\vspace{1em}

The foreign keys for their respective tables are:

-Date - None

-Country - None

-Virus - $dateRep$ and $geoId$

\vspace{1em}

I established any partial key dependencies, $day, month$ and $year$ were dependent on $dateRep$ and not $geoId$, so they were only partially dependent on the key.

$countriesAndTerritories, countryTerritoryCode$ and $continentExp$ were dependent on $geoId$ and not $dateRep$, so they were only partially dependent on the key. I then divided the attributes so that they all depended on the entirety of their respective keys.

\subsection*{EX7}

As the relations currently exist, there are no transitive dependencies.

$countriesAndTerritories$ and $countryterritoryCode$ would be a transitive dependency as\newline
$geoId \rightarrow countriesAndTerritories$ and $countriesAndTerritories \rightarrow countryterritoryCode$.

But this is not a valid transitive dependency as $countriesAndTerritories \rightarrow geoId$, and for a transitive dependency $A \rightarrow B \rightarrow C$, there cannot exist $B \rightarrow A$.

\subsection*{EX8}

The relations in 3NF:

\begin{center}
\begin{tabular}{ |c|c|c|c| } 
 \hline
 \multicolumn{4}{|c|}{Date} \\
 \hline
 dateRep & day & month & year \\
 \hline
 DATE & INTEGER & INTEGER & INTEGER \\
 \hline
\end{tabular}
\end{center}

\begin{center}
\begin{tabular}{ |c|c|c|c|c| } 
 \hline
 \multicolumn{5}{|c|}{Country} \\
 \hline
 geoId & countriesAndTerritories & countryterritoryCode & continentExp & popData2018 \\
 \hline
 TEXT & TEXT & TEXT & TEXT & INTEGER \\
 \hline
\end{tabular}
\end{center}

\begin{center}
\begin{tabular}{ |c|c|c|c| } 
 \hline
 \multicolumn{4}{|c|}{Virus Info} \\
 \hline
 dateRep & geoId & cases & deaths \\
 \hline
 TEXT & TEXT & INTEGER & INTEGER \\
 \hline
\end{tabular}
\end{center}

% \begin{center}
% \begin{tabular}{ |c|c|c|c| } 
%  \hline
%  \multicolumn{4}{|c|}{Date} \\
%  \hline
%  dateRep & day & month & year \\
%  \hline
%  DATE & INTEGER & INTEGER & INTEGER \\
%  \hline
% \end{tabular}
% \end{center}

% \begin{center}
% \begin{tabular}{ |c|c|c|c| } 
%  \hline
%  \multicolumn{4}{|c|}{Virus Info} \\
%  \hline
%  dateRep & geoId & cases & deaths \\
%  \hline
%  DATE & CHAR(N) & INTEGER & INTEGER \\
%  \hline
% \end{tabular}
% \end{center}

% \begin{center}
% \begin{tabular}{ |c|c| } 
%  \hline
%  \multicolumn{2}{|c|}{Country Code} \\
%  \hline
%  geoId & countryterritoryCode\\
%  \hline
%  CHAR(n) & CHAR(n) \\
%  \hline
% \end{tabular}
% \end{center}

% \begin{center}
% \begin{tabular}{ |c|c|c|c| } 
%  \hline
%  \multicolumn{4}{|c|}{Country Info} \\
%  \hline
%  geoId & countriesAndTerritories & continentExp & popData2018\\
%  \hline
%  CHAR(n) & VARCHAR(n) & VARCHAR(n) & BIGINT \\
%  \hline
% \end{tabular}
% \end{center}

Since there are no transitive dependencies, the relations are the same as they were in 2NF.

\vspace{1em}

The primary keys for their respective tables are:

-Date - $dateRep$

-Country - $geoId$

-Virus - $dateRep, geoId$

\vspace{1em}

The foreign keys for their respective tables are:

-Date - None

-Country - None

-Virus - $dateRep$ and $geoId$

% The keys for the respective tables are:

% -Date - $dateRep$

% -Virus Info - $dateRep, geoId$

% -Country Code - $geoId$

% -Country Info - $geoId$

\subsection*{EX9}

The relation is in Boyce-Codd Normal Form (BCNF), BCNF requires for every all dependencies $A \rightarrow B$, $A$ must be a superkey.

For example this is true in Country, $countriesAndTerritories$ determines other attributes, though it is a superkey so it is allowed in BCNF.

\pagebreak

\section*{Modelling}

\subsection*{EX11}

The indexes I have created in their respective tables are:

-Date - Index $dateIndex$ created on $dateRep$

-Virus - Index $dateLocation$ created on $dateRep$ and $geoId$

-Country - Index $locationName$ created on $countriesAndTerritories$ and index $locationId$ created on $geoId$

I chose $dateRep$ as the index for Date as anyone querying the Date table is likely to be doing so using a specific date.

I chose $dateRep$ and $geoId$ to be the indexes for Virus as case and death information is likely to be queried with a time and place.

I chose $countriesAndTerritories$ as an index for Country as country information is likely to be queried with the country name.

I also chose $geoId$ as an index for Country as country information is also likely to be queried using the countries ID.

\section*{Querying}

\subsection*{EX14}

\begin{lstlisting}[
           language=SQL,
           showspaces=false,
           basicstyle=\ttfamily,
           numbers=left,
           numberstyle=\tiny,
           commentstyle=\color{gray},
           keywords={[2]{}}
        ]
SELECT sum(cases) AS "total cases", 
       sum(deaths) AS "total deaths"
FROM Virus;
\end{lstlisting}

\subsection*{EX15}

\begin{lstlisting}[
           language=SQL,
           showspaces=false,
           basicstyle=\ttfamily,
           numbers=left,
           numberstyle=\tiny,
           commentstyle=\color{gray},
           keywords={INNER, SELECT, FROM, ORDER, BY, WHERE, ON, ASC, DESC, AS, JOIN, GROUP},
           keywords={[2]{}}
        ]
SELECT virus.dateRep AS date,
       virus.cases AS "number of cases"
FROM Virus INNER JOIN date ON virus.dateRep = date.dateRep
WHERE virus.geoId = 'UK'
ORDER BY date.year, date.month, date.day ASC;
\end{lstlisting}

\pagebreak

\subsection*{EX16}

\begin{lstlisting}[
           language=SQL,
           showspaces=false,
           basicstyle=\ttfamily,
           numbers=left,
           numberstyle=\tiny,
           commentstyle=\color{gray},
           keywords={INNER, SELECT, FROM, ORDER, BY, WHERE, ON, ASC, DESC, AS, JOIN, GROUP},
           keywords={[2]{}}
        ]
SELECT country.continentExp AS continent,
       virus.dateRep AS date,
       sum(virus.cases) AS "number of cases",
       sum(virus.deaths) AS "number of deaths"
FROM Virus
INNER JOIN country ON virus.geoId = country.geoId
INNER JOIN date ON virus.dateRep = date.dateRep
GROUP BY country.continentExp, virus.dateRep
ORDER BY date.year, date.month, date.day ASC;
\end{lstlisting}

\subsection*{EX17}

\begin{lstlisting}[
           language=SQL,
           showspaces=false,
           basicstyle=\ttfamily,
           numbers=left,
           numberstyle=\tiny,
           commentstyle=\color{gray},
           keywords={INNER, SELECT, FROM, ORDER, BY, WHERE, ON, ASC, DESC, AS, JOIN, GROUP},
           keywords={[2]{}}
        ]
SELECT country.countriesAndTerritories AS country,
       round((sum(virus.cases) * 1.0 / country.popData2018) * 100, 2) AS "% cases of population",
       round((sum(virus.deaths) * 1.0 / country.popData2018) * 100, 2) AS "% cases of deaths"
FROM Virus
INNER JOIN Country ON virus.geoId = country.geoId
GROUP BY country.countriesAndTerritories, country.popData2018;
\end{lstlisting}

\subsection*{EX18}

\begin{lstlisting}[
           language=SQL,
           showspaces=false,
           basicstyle=\ttfamily,
           numbers=left,
           numberstyle=\tiny,
           commentstyle=\color{gray},
           keywords={INNER, SELECT, FROM, ORDER, BY, WHERE, ON, ASC, DESC, AS, JOIN, GROUP, LIMIT},
           keywords={[2]{}}
        ]
SELECT country.countriesAndTerritories AS "country name",
       round((sum(virus.deaths) * 1.0 / sum(virus.cases)) * 100, 2) AS "% cases of deaths"
FROM Virus
INNER JOIN Country ON virus.geoId = country.geoId
GROUP BY country.countriesAndTerritories, country.popData2018
ORDER BY "% cases of deaths" DESC
LIMIT 10;
\end{lstlisting}

\pagebreak

\subsection*{EX19}

\begin{lstlisting}[
           language=SQL,
           showspaces=false,
           basicstyle=\ttfamily,
           numbers=left,
           numberstyle=\tiny,
           commentstyle=\color{gray},
           keywords={INNER, SELECT, FROM, ORDER, BY, WHERE, ON, ASC, DESC, AS, JOIN, GROUP, LIMIT},
           keywords={[2]{}}
        ]
SELECT date.dateRep,
       sum(virus.cases) OVER (
           ORDER BY date.year, date.month, date.day
           ROWS BETWEEN
               UNBOUNDED PRECEDING
               AND CURRENT ROW
           ) AS 'cumulative UK cases',
       sum(virus.deaths) OVER (
           ORDER BY date.year, date.month, date.day
           ROWS BETWEEN
               UNBOUNDED PRECEDING
               AND CURRENT ROW
           ) AS 'cumulative UK deaths'
FROM Date
INNER JOIN Virus ON date.dateRep = virus.dateRep AND virus.geoId = 'UK'
ORDER BY date.year, date.month, date.day ASC;
\end{lstlisting}


\end{document}